\documentclass[12pt,a4paper]{article}
\usepackage[NoDate]{currvita}
\usepackage{mflogo}
\usepackage{hyperref}

\pagenumbering{gobble}

\begin{document}
\begin{cv}{Jure Zbontar}

\begin{cvlist}{}
\item[Email] jure.zbontar@gmail.com
\item[Telephone] (917) 633-9170
\item[Born] May 28, 1985
\item[Citizenship] Slovenian

\end{cvlist}

\subsubsection*{Employment}
\begin{cvlist}{}
\item[2017 - present] \emph{Research Engineer} \\
Facebook AI Research
\end{cvlist}

\subsubsection*{Education}
\begin{cvlist}{}

\item[2016 - 2017] \emph{Postdoctoral Associate} \\
Courant Institute of Mathematical Sciences \\
New York University

\item[2008 - 2016] \emph{PhD in Machine Learning} \\
Faculty of Computer and Information Science \\
University of Ljubljana, Slovenia \\
GPA: 9.909 / 10

\item[2014 - 2016] \emph{Visiting Student} \\
Courant Institute of Mathematical Sciences \\
New York University

\item[2004 - 2008] \emph{BSc in Computer Science and Mathematics} \\
Faculty of Computer and Information Science \\
University of Ljubljana, Slovenia \\
GPA: 9.538 / 10

\end{cvlist}

\paragraph{Research Visit}
I visited Yann LeCun's group at New York University where I worked on deep
convolutional networks for stereo vision. Our method ranked first in terms of
accuracy on two widely used stereo datasets,
\href{http://www.cvlibs.net/datasets/kitti/eval_stereo.php}{KITTI} and
\href{http://vision.middlebury.edu/stereo/eval3/}{Middlebury}. We described our
method in two papers, \textit{Computing the Stereo Matching Cost with a
Convolutional Neural Network}, presented at CVPR 2015, and \textit{Stereo
Matching by Training a Convolutional Neural Network to Compare Image Patches},
published in JMLR. The work was orally presented at ECCV 2014. The source code
is available online: \url{https://github.com/jzbontar/mc-cnn}.

\paragraph{Machine Learning Competitions}
I participated in several machine learning competitions on Kaggle and Tunedit.

\begin{cvlist}{}
\item[2013 \quad 5th / 249] The Marinexplore and Cornell University Whale Detection 
Challenge, \\
\url{http://www.kaggle.com/c/whale-detection-challenge}
\item[2012 \quad 1st / 126] Topical Classification of Biomedical Research Papers, 
JRS 2012 Data Mining Competition, \\
\url{http://tunedit.org/challenge/JRS12Contest}
\item[2012 \quad 2nd / 91] EMC Israel Data Science Challenge, \\
\url{http://www.kaggle.com/c/emc-data-science}
\item[2012 \quad 2nd / 156] The Hewlett Foundation: Short Answer Scoring 
Competition, \\
\url{http://www.kaggle.com/c/asap-sas}
\item[2011 \quad 1st / 16] Algorithm for Optimal Job Scheduling and Task 
Allocation under Constraints \\
\url{http://tunedit.org/challenge/job-scheduling}
\item[2010 \quad 1st / 22] Forecast Eurovision Voting, \\
\url{http://www.kaggle.com/c/Eurovision2010}
\end{cvlist}

\paragraph*{Programming Competitions}
During my undergraduate years I entered several programming competitions. The
lessons learned had a great impact on my programming style and my way of
thinking and solving problems.

\begin{cvlist}{}
\item[2008] ACM Central European Regional Contest
\item[2006 - 2008] ACM Slovenian Regional Contest
\end{cvlist}

\paragraph*{Articles Mentioning My Work}
Jennifer Carpenter wrote \textit{May the Best Analyst Win} featuring my win on the
Eurovision Voting competition hosted by Kaggle. Published in Science (February
2011, Vol.  331, Issue 6018).

\paragraph*{Coding}
I have experience with several programming languages: C, Java, Go, LISP,
Haskell, OCaml, Erlang, Lua, Cuda, and Prolog. I write most of my code in Python with
PyTorch.

\newpage
\nocite{zbontar2015computing, vzbontar2012team, vzbontar2016stereo, demvsar2013orange, provodin2016fast, zbontar2012short, tygert2018compressed, zbontar2018fastmri, sriram2020grappanet, knoll2020advancing, knoll2020fastmri, sriram2020end, recht2020using, jing2020implicit, tygert2020simulating}
\bibliography{zbontar}{}
\bibliographystyle{unsrt}

\end{cv}
\end{document}
