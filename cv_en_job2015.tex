\documentclass[12pt,a4paper]{article}
\usepackage[NoDate]{currvita}
\usepackage{mflogo}
\usepackage{hyperref}

\begin{document}
\begin{cv}{Jure \v{Z}bontar}

\begin{cvlist}{}
\item[Email] \href{mailto:jure.zbontar@gmail.com}{jure.zbontar@gmail.com}

\item[Born] May 28, 1985
\item[Citizenship] Slovenian

\end{cvlist}

\begin{cvlist}{Education}
\item[2008 - present] PhD student (will defend in 2016) \\
Faculty of Computer and Information Science \\
University of Ljubljana, Slovenia \\
GPA: 9.909 / 10

\item[2004 - 2008] BSc in Computer Science and Mathematics \\
Faculty of Computer and Information Science \\
University of Ljubljana, Slovenia \\
GPA: 9.538 / 10

\end{cvlist}

\subsection*{Research Visit}
I visited Yann LeCun's group at New York University in January 2014
where I spent 7 months working on convolutional neural networks for
stereo vision. The system we developed was the best performing method on
the KITTI stereo dataset at the time. The paper ``Computing the Stereo
Matching Cost with a Convolutional Neural Network" was presented at
CVPR 2015.

\subsection*{Machine Learning Competitions}
Machine learning competitions are perfect for testing new ideas and
comparing them to established methods. I have learned many valuable
lessons about how to make learning algorithm behave well on real datasets.

\begin{cvlist}{}
\item[2013 \quad 5th / 249] The Marinexplore and Cornell University Whale Detection 
Challenge\footnotemark[1], \\
\url{http://www.kaggle.com/c/whale-detection-challenge}
\item[2012 \quad 1st / 126] Topical Classification of Biomedical Research Papers, 
JRS 2012 Data Mining Competition\footnotemark[2], \\
\url{http://tunedit.org/challenge/JRS12Contest}
\item[2012 \quad 2nd / 91] EMC Israel Data Science Challenge\footnotemark[2], \\
\url{http://www.kaggle.com/c/emc-data-science}
\item[2012 \quad 2nd / 156] The Hewlett Foundation: Short Answer Scoring 
Competition\footnotemark[1], \\
\url{http://www.kaggle.com/c/asap-sas}
\item[2011 \quad 1st / 16] Algorithm for Optimal Job Scheduling and Task 
Allocation under Constraints\footnotemark[1] \\
\url{http://tunedit.org/challenge/job-scheduling}
\item[2010 \quad 1st / 22] Forecast Eurovision Voting\footnotemark[1], \\
\url{http://www.kaggle.com/c/Eurovision2010}
\end{cvlist}

\footnotetext[1]{Entered competition alone.}
\footnotetext[2]{Team leader.}
 
\subsection*{Programming Competitions}
During my undergraduate years I enjoyed solving algorithmic problems. I entered
as many programming competitions as I possibly could. The lessons learned
had a great impact on my programming style and my way of thinking and solving problems.

\begin{cvlist}{}
\item[2008] ACM Central European Regional Contest
\item[2006 - 2008] ACM Slovenian Regional Contest
\end{cvlist}

\subsection*{Programming Skills}
I really like programming. During my career I tried many different
programming languages from C, C++, Java and Go to LISP, Haskell, OCaml,
Erlang and Prolog. I also tried web programming with JavaScript, PHP and
ActionScript. Today, most of my code is written in Python (with NumPy,
Theano and scikit-learn) and Lua (with Torch). If my code is not running fast enough,
I like to speed it up with CUDA or C.

\end{cv}
\end{document}
