\documentclass[12pt,a4paper]{article}
\usepackage[NoDate]{currvita}
\usepackage{mflogo}
\usepackage{hyperref}

\pagenumbering{gobble}

\begin{document}
\begin{cv}{Jure \v{Z}bontar}

\begin{cvlist}{}
\item[Email] \href{mailto:jure.zbontar@gmail.com}{jure.zbontar@gmail.com}
\item[Telephone] +1-917-633-9170
\item[Born] May 28, 1985
\item[Citizenship] Slovenian

\end{cvlist}

\subsubsection*{Education}
\begin{cvlist}{}
\item[2008 - 2016] \emph{PhD in Machine Learning} \\
Faculty of Computer and Information Science \\
University of Ljubljana, Slovenia \\
GPA: 9.909 / 10

\item[2014 - 2016] \emph{Visiting Student} \\
New York University \\
Courant Institute of Mathematical Sciences

\item[2004 - 2008] \emph{BSc in Computer Science and Mathematics} \\
Faculty of Computer and Information Science \\
University of Ljubljana, Slovenia \\
GPA: 9.538 / 10

\end{cvlist}

\paragraph{Research Visit}
I visited Yann LeCun's group at New York University in 2014 and 2015 where I
spent 12 months working on deep convolutional neural networks for stereo
vision. The system we developed was the best performing method on the
KITTI\footnote{\url{http://www.cvlibs.net/datasets/kitti/eval_stereo.php}} and
Middlebury\footnote{\url{http://vision.middlebury.edu/stereo/eval3/}} stereo
datasets at the time. We described our approach in two papers: ``Computing the
Stereo Matching Cost with a Convolutional Neural Network", which was presented
at CVPR 2015, and ``Stereo Matching by Training a Convolutional Neural Network
to Compare Image Patches'', which was published in JMLR. The source code of our
method is available online\footnote{\url{http://github.com/jzbontar/mc-cnn}}.

\paragraph{Machine Learning Competitions}
Machine learning competitions are perfect for testing new ideas and comparing
them to established methods. I have learned many valuable lessons about how to
make learning algorithm behave well on real datasets.

\begin{cvlist}{}
\item[2013 \quad 5th / 249] The Marinexplore and Cornell University Whale Detection 
Challenge, \\
\url{http://www.kaggle.com/c/whale-detection-challenge}
\item[2012 \quad 1st / 126] Topical Classification of Biomedical Research Papers, 
JRS 2012 Data Mining Competition, \\
\url{http://tunedit.org/challenge/JRS12Contest}
\item[2012 \quad 2nd / 91] EMC Israel Data Science Challenge, \\
\url{http://www.kaggle.com/c/emc-data-science}
\item[2012 \quad 2nd / 156] The Hewlett Foundation: Short Answer Scoring 
Competition, \\
\url{http://www.kaggle.com/c/asap-sas}
\item[2011 \quad 1st / 16] Algorithm for Optimal Job Scheduling and Task 
Allocation under Constraints \\
\url{http://tunedit.org/challenge/job-scheduling}
\item[2010 \quad 1st / 22] Forecast Eurovision Voting, \\
\url{http://www.kaggle.com/c/Eurovision2010}
\end{cvlist}

\paragraph*{Programming Competitions}
During my undergraduate years I enjoyed solving algorithmic problems. I entered
as many programming competitions as I could. The lessons learned had a great
impact on my programming style and my way of thinking and solving problems.

\begin{cvlist}{}
\item[2008] ACM Central European Regional Contest
\item[2006 - 2008] ACM Slovenian Regional Contest
\end{cvlist}

\paragraph*{Articles Mentioning My Work}
Jennifer Carpenter wrote ``May the Best Analyst Win'' featuring my win on the
Eurovision Voting competition hosted by Kaggle. Published in Science (February
2011, Vol.  331, Issue 6018).

\paragraph*{Programming Skills}
I enjoy programming. I experimented with many programming languages
throughout my career: from C, C++, Java and Go to LISP, Haskell, OCaml, Erlang
and Prolog. Today, most of my code is written in Lua with Torch. If my code is
not running fast enough, I use CUDA or C to speed it up.

\end{cv}
\end{document}
