\documentclass[12pt,a4paper]{article}
\usepackage[NoDate]{currvita}
\usepackage{mflogo}
\usepackage{hyperref}

\renewcommand\refname{Publications}

\begin{document}
\begin{cv}{Jure \v{Z}bontar}
\noindent \emph{Curriculum Vitae}

\begin{cvlist}{}
\item[Address] Jure \v{Z}bontar \\
Raku\v{s}eva 26 \\
1000 Ljubljana \\
Slovenia

\item[Telephone] +1 917-633-9170

\item[Email] jure.zbontar@gmail.com

\item[Born] May 28, 1985, in Ljubljana, Slovenia
\item[Citizenship] Slovenian

\end{cvlist}

\begin{cvlist}{Education}
\item[2008 - present] PhD student \\
Faculty of Computer and Information Science \\
University of Ljubljana \\
Slovenia \\
GPA: 9.909 / 10

\item[2004 - 2008] BSc in Computer Science and Mathematics \\
Faculty of Computer and Information Science \\
University of Ljubljana \\
Slovenia \\
GPA: 9.538 / 10

\end{cvlist}

\begin{cvlist}{Professional Positions}
\item[2015 - present] Junior Research Scientist \\
Courant Institute of Mathematical Sciences \\
New York University

\item[2010 - 2015] Teaching Assistant, Bioinformatics Laboratory  \\
Faculty of Computer and Information Science \\
University of Ljubljana

\item[2008 - 2010] Teaching Assistant, Artificial Intelligence Laboratory \\
Faculty of Computer and Information Science \\
University of Ljubljana

\item[2007 - 2008] Information Systems Laboratory \\
Faculty of Computer and Information Science \\
University of Ljubljana

\end{cvlist}

\begin{cvlist}{Research Visits}
\item[2014] Courant Institute of Mathematical Sciences \\
New York University \\
Mentor: Yann LeCun (\href{mailto:yann@cs.nyu.edu}{yann@cs.nyu.edu})
\end{cvlist}

\begin{cvlist}{Consulting}
\item[2013 - 2014] CTB/McGraw-Hill \\
Implement system for automatic short answer scoring \\
\url{https://bitbucket.org/jzbontar/asap} \\
Contact: Michelle Barrett (\href{mailto:michelle_barrett@ctb.com}{michelle\_barrett@ctb.com})
\end{cvlist}

\subsection*{Programming Competitions}
During my undergraduate years I enjoyed solving algorithmic problems. I
entered as many programming competitions as I could. The lessons learned
had a great impact on my programming style and my way of thinking and
solving problems.

\begin{cvlist}{}
\item[2009] Open Krakow Team Programming Challenge
\item[2008 - 2010] Entered Many TopCoder Competitions
\item[2008] ACM Central European Regional Contest
\item[2006 - 2008] ACM Slovenian Regional Contest
\item[2006 - 2010] Spent a lot of time on \url{http://uva.onlinejudge.org/} online judge
\end{cvlist}

\subsection*{Machine Learning Competitions}
Machine learning competitions are perfect for testing new ideas and
comparing them to established methods. I have learned many valuable
lessons about how to make learning algorithm behave well on real
datasets. It's really exciting to see the machine learning approach beat
human benchmarks and hand engineering.

\begin{cvlist}{}
\item[2013 \quad 5th / 249] The Marinexplore and Cornell University Whale Detection
Challenge\footnotemark[1], \\
\url{http://www.kaggle.com/c/whale-detection-challenge}
\item[2012 \quad 1st / 126] Topical Classification of Biomedical Research Papers,
JRS 2012 Data Mining Competition\footnotemark[2], \\
\url{http://tunedit.org/challenge/JRS12Contest}
\item[2012 \quad 2nd / 91] EMC Israel Data Science Challenge\footnotemark[2], \\
\url{http://www.kaggle.com/c/emc-data-science}
\item[2012 \quad 2nd / 156] The Hewlett Foundation: Short Answer Scoring
Competition\footnotemark[1], \\
\url{http://www.kaggle.com/c/asap-sas}
\item[2011 \quad 1st / 16] Algorithm for Optimal Job Scheduling and Task
Allocation under Constraints\footnotemark[1] \\
\url{http://tunedit.org/challenge/job-scheduling}
\item[2010 \quad 1st / 22] Forecast Eurovision Voting\footnotemark[1], \\
\url{http://www.kaggle.com/c/Eurovision2010}
\end{cvlist}

\footnotetext[1]{Entered competition alone.}
\footnotetext[2]{Team leader.}

\subsection*{Completed Online Courses}
The proliferation of online courses has definitely played a role in my
education.  The quality of some of the courses is absolutely amazing. In
the past few years I have completed the following online courses:

\begin{cvlist}{}
\item[Coursera] 
\begin{itemize}
\item Machine Learning (Andrew Ng)
\item Neural Networks for Machine Learning (Geoffrey Hinton)
\item Probabilistic Graphical Models (Daphne Koller)
\item Writing in the Sciences (Kristin Sainani)
\end{itemize}

\item[Udacity] 
\begin{itemize}
\item Introduction to Parallel Programming (John Owens, David Luebke)
\item Introduction to Artificial Intelligence (Sebastian Thrun, Peter Norvig)
\end{itemize}

\item[Stanford] 
\begin{itemize}
\item CS229: Machine Learning (Andrew Ng)
\item CS294A: Deep Learning and Unsupervised Feature Learning (Andrew Ng)
\end{itemize}

\item[Caltech] 
\begin{itemize}
\item Learning From Data (Yaser Abu-Mostafa)
\end{itemize}

\end{cvlist}

\subsection*{Programming Skills}
I really like programming. During my career I tried many different
programming languages from C, C++, Java and Go to LISP, Haskell, OCaml,
Erlang and Prolog. I also tried web programming with JavaScript, PHP and
ActionScript. Today, most of my code is written in Python (with NumPy,
Theano and scikit-learn) and Lua (with Torch). If my code is not running fast enough,
I like to speed it up with Cython, CUDA or C.
\end{cv}

\nocite{vzbontar2014computing,demvsar2013orange,zbontar2012team}
\bibliographystyle{plain}{}
\bibliography{zbontar}

\end{document}
